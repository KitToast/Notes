\documentclass{../quantum.tex}
\begin{document}
\section{Qubits}

A {\bf qubit} $\ket{\Psi}$ is the linear combination of basis elements $\ket{0}$ and $\ket{1}$ interpreted as a superposition of $0,1$:
$$ \ket{\Psi} = \alpha \ket{0} + \beta \ket{1}, \quad \alpha,\beta \in \mathbb{C}$$

Taking $n$ tensor products of $\ket{0}$,$\ket{1}$ yields entangled states of $n$-qubits:
\begin{gather*}
  \ket{0} \otimes \ket{0} ... \otimes \ket{0} = \ket{0....0} \\   \ket{0} \otimes \ket{0} ... \otimes \ket{1} = \ket{0....1} \\
  ... \\
    \ket{1} \otimes \ket{1} ... \otimes \ket{1} = \ket{1....1}
\end{gather*}
Let $\mathbb{C}^2$ be the 2 dimensional $\mathbb{C}$-vector space representing the space of superpostions of a single qubit. Then the $n$-qubit $\mathcal{H}_n$ can be represented as:
$$\mathcal{H}_n = (\mathbb{C}^2)^{\otimes n}  $$ In other words, if $\ket{\Psi} \in \mathcal{H}_n$, then
$$ \ket{\Psi} = \sum_{i= 0}^{2^n -1} c_i \ket{i} \quad c_i \in \mathbb{C} $$
by definition of the $n$-tensor product of the 2-dimensional complex vector space.
\end{document}
