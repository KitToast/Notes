\documentclass{../quantum.tex}

\begin{document}
\section{Quantum Mechanics}

\subsection{Density Matrices}

Suppose we have an ensemble of states\{\ket{\psi_1},...,\ket{\psi_k}\} in a statisitcal mixture in the sense that we know the distribution of the states. Let $p_i$ denote the probability of selecting state $\ket{\psi_i}$ for $1 \leq i \leq k$ where $\sum_i p_i = 1$.

We wish to find a way to calculate the expected value of any observable $A$ on this ensemble. Note that the linear superposition of these states in the form $\sum_{i=1}^k p_i \ket{\psi_i}$ is generally not the same as a statistical mixture of the individual states $\ket{\psi_i}$.

From Postulate three, we can calculate the probability that measuring with $A = \sum_i a_iP_i$ will yield outcome $a_j$:

$$ p(j) = \sum_{i=1}^k p_i \bra{\psi_i}P_j\ket{\psi_i} $$

Note that the $\bra{\psi_i}P_j\ket{\psi_i}$ represent the probability of obtaining outcome $a_j$ given the state vector is $\ket{\psi_i}$ as per postulate three.

\end{document}
