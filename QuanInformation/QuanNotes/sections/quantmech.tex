\documentclass{../quantum.tex}

\begin{document}
\section{Quantum Mechanics}

\subsection{Postulates of Quantum Mechanics}

\subsection{Density Matrices}

Suppose we have an ensemble of states $\{\ket{\psi_1},...,\ket{\psi_k}\}$ in a statisitcal mixture in the sense that we know the distribution of the states. Let $p_i$ denote the probability of selecting state $\ket{\psi_i}$ for $1 \leq i \leq k$ where $\sum_i p_i = 1$.

We wish to find a way to calculate the expected value of any observable $A$ on this ensemble. Note that the linear superposition of these states in the form $\sum_{i=1}^k p_i \ket{\psi_i}$ is generally not the same as a statistical mixture of the individual states $\ket{\psi_i}$.

From Postulate three, we can calculate the probability that measuring with $A = \sum_i a_iP_i$ will yield outcome $a_j$:

$$ p(j) = \sum_{i=1}^k p_i \bra{\psi_i}P_j\ket{\psi_i} $$

Note that the $\bra{\psi_i}P_j\ket{\psi_i}$ represent the probability of obtaining outcome $a_j$ given the state vector is $\ket{\psi_i}$ as per postulate three. Thus, our expected value for the observable $A$ would objects:
$$ \langle A \rangle = \sum_{j=1}^n a_j p(j) = \sum_{j=1}^n a_i (\sum_{i=1}^k p_i \bra{\psi_i}P_j\ket{\psi_i}) = \sum_{i=1}^k p_i \bra{\psi_i}\sum_{i=1}^n a_j P_j\ket{\psi_i} = \sum_{i=1}^kp_i \bra{\psi_i}A\ket{\psi_i} $$

Define the {\bf density matrix} of the ensemble to be
$$ \rho = \sum_{i=1}^k p_i \ket{\psi_i}\bra{\psi_i} $$

and $\langle A \rangle = Tr(\rho A)$. We verify this as follows:

\begin{gather}
 Tr(\rho A) = \sum_{j=1}^n \bra{j}\rho A \ket{j} = \sum_{j=1}^n \bra{j} (\sum_{i=1}^k p_i\ket{\psi_i} \bra{\psi_i})A\ket{j} = \sum_{j=1}^n \sum_{i=1}^k p_i \bra{j}\ket{\psi_i}\bra{\psi_i}A\ket{j} = \\
 \sum_{i=1}^k p_i\bra{\psi_i}A(\sum_{j=1}^n \ket{j}\bra{j}) \ket{\psi_i} = \sum_{i=1}^k p_i \bra{\psi_i}A\ket{\psi_i} = \langle A \rangle
\end{gather}
where the next-to-last equality stems from the completeness criterion: $\sum_i \ket{i}\bra{i} = I$. Thus, we have a method to describe expectation and probabilities in terms of operators rather than state vectors. Note that a pure state can be trivially described through the density matrix formalulation by $\rho = \ket{\psi}\bra{\psi}$.

The density matrix has the following properties: Let $\{\ket{i}\}$ denote an orthogonal basis of our Hilbert space $\mathcal{H}$:

\begin{enumerate}
  \item {\bf $\rho$ is Hermitian}: Let
  $$ \rho_{ij} = \bra{i}\rho\ket{j} = \sum_{m=1}^k p_m \bra{i}\ket{\psi_m} \bra{\psi_m}\ket{j} = \sum_{m=1}^k p_m c_i^{(m)} (c_j^{(m)})^*$$ Thus,
  $$ \rho_{ji}^* = \sum_{m=1}^k p_m (c_j^{(m)})^* c_i^{(m)} = \rho_{ij}  $$
  \item {\bf $\rho$ has unit trace} By our formula above:
  $$ Tr(\rho) = \sum_{i=1}^n \rho_{ii} = \sum_{i=1}^n\sum_{m=1}^k p_m c_{i}^{(m)}(c_{i}^{(m)})^* = \sum_{m=1}^k p_m \sum_{i=1}^n|c_i^{(m)}|^2 =
  \sum_{m=1}^k p_m = 1
  $$
  \item {\bf $\rho$ is non-negative}: Let $\ket{\phi} \in \mathcal{H}$. It suffices to prove that $\bra{\phi}\rho\ket{\phi} \geq 0$.
\end{enumerate}

Fromm property one above, we derive the following criteria for determining if a given density matrix $\rho$ represents a mixed state or a pure state:

\begin{theorem}
  Given density matrix $\rho$, $Tr(\rho^2) = 1$ iff $\rho$ represents a pure state. On the other hand, $Tr(\rho^2) < 1$ iff $\rho$ represents a mixed state.
\end{theorem}

\begin{proof}

\end{proof}

\subsubsection{How to prepare $\rho$}
Suppose we have an ensemble of the form $\{p(i),\ket{\psi_i}\}$ where $p$ refers to a distribution of probabilities over the indices $\{i\}$. We can classically prepare such a state by tossing a biased die obeying the distribution above, and preparing the corresponding state based on the outcome.

\subsection{Composite Systems}

We can use the density matrix formalism to express statistical measurements in subsystems in the following sense:

Let $\mathcal{H} = \mathcal{H}_1 \otimes \mathcal{H}_2$ be the \textit{bipartite} separable system consisting of two subsystems expressed as Hilbert spaces $\mathcal{H}_1,\mathcal{H}_2$. By definition of the tensor product if $\ket{\psi} \in \mathcal{H}$, $\ket{\psi}$ has the form:
$$ \ket{\psi} = \sum_{i,\alpha} c_{i}c_{\alpha} \ket{i}\otimes\ket{\alpha} =  \sum_{i,\alpha} c_{i,\alpha} \ket{i}\otimes\ket{\alpha}$$
if we let $\{\ket{i}\},\{\ket{\alpha}\}$ be sets of orthogonal basis vectors in $\mathcal{H}_1,\mathcal{H}_2$ respectively.

By our definiton in the previous section, the density matrix for our pure state $\ket{\psi}$ will be:

$$ \rho = \ket{\psi}\bra{\psi} = \sum_{i,\alpha}\sum_{j,\beta} c_{i,\alpha}(c_{j,\beta})^*\ket{i}\ket{\alpha}\bra{j}\bra{\beta} = \sum_{(i,j),(\alpha,\beta)} \lambda_{(i,j),(\alpha,\beta)} \ket{i}\bra{j}\otimes \ket{\alpha}\bra{\beta}$$

Suppose we have an observable $A_1$ for subsystem $\mathcal{H}_1$, and we wish to find $\langle A_1 \rangle$ given $\mathcal{H}$. Let $A_1 \otimes I_2: \mathcal{H} \rightarrow \mathcal{H}$ be the unique linear map induced by $A_1:\mathcal{H}_1 \rightarrow \mathcal{H}_1$ and $I_2: \mathcal{H}_2 \rightarrow \mathcal{H}_2$.

\begin{gather}
  \langle A_1 \otimes I_2 \rangle = Tr(\rho(A_1 \otimes I_2)) = \sum_{i,\alpha} \bra{i}\bra{\alpha}\rho(A_1 \otimes I_2) \ket{i}\ket{\alpha} = \\
  \sum_{i,\alpha} \bra{i}\bra{\alpha}[
  \sum_{(k,l),(\beta,\gamma)} \lambda_{(k,l),(\beta,\gamma)} \ket{k}\bra{l}\otimes \ket{\beta}\bra{\gamma}]
  )(A_1 \otimes I_2)\ket{i}\ket{\alpha} = \\
  \sum_{i} \bra{i} \sum_{\alpha} \bra{\alpha}(\sum_{(k,l),(\beta,\gamma)}
   \lambda_{(k,l),(\beta,\gamma)} \ket{k}\bra{l}\otimes \ket{\beta}\bra{\gamma})\ket{\alpha} A_i \ket{i} = \sum_{i} \bra{i} = \\ Tr_2(\rho)A_i \ket{i} = Tr(A_iTr_b(\rho))
\end{gather}

where $$ \rho_1= Tr_2(\rho) = \sum_{\alpha} \bra{\alpha} \rho \ket{\alpha} $$ is the \textit{partial trace} of density operator $\rho$. Note that these equalties follow directly from the bilinear property of the tensor product on Hilbert spaces $\mathcal{H}_1,\mathcal{H}_2$.

Similarly, for $\mathcal{H}_1$:
$$\rho_2 = Tr_1(\rho) = \sum_{i} \bra{i}\rho\ket{i} $$

It can be directly checked that $\rho_m$ for $m = 1,2$ is indeed a density operator in that it is a Hermitian semi-definite operator with unit trace. We refer to the $\rho_m$ as \textit{local density operators} as they refer to density operators in respect to the individual subsystems.

The local density operators above can be seen as the quantim analogue of the marginal distribution defined in classical probability theory i.e if $p_{X,Y}(x,y)$ is a joint probability distribution over random variables $X,Y$, then the marginal distribution for $X$ is
$p_X(x) = \sum_{y} p_{X,Y}(x,y)$


\subsection{Separable States}






\end{document}
