\documentclass[quantum.tex]{subfiles}

\begin{document}
\section{Quantum Gates}

Quantum gates are esstentially unitary operators on Hilbert spaces. These unitary operators act on single qubits by rotating them along the Bloch sphere.

\subsection{Hadamard Gates}
There are many gates that do not have classical analouges. One example is the Hadamard gate $H: \mathbb{C}^2 \rightarrow \mathbb{C}^2$
\begin{gather*}
  \ket{0} \mapsto \frac{1}{\sqrt{2}}(\ket{0} + \ket{1}) \\
  \ket{1} \mapsto \frac{1}{\sqrt{2}}(\ket{0} - \ket{1})
\end{gather*}
which is represented as the matrix:
$$ H =  \frac{1}{\sqrt{2}} \begin{bmatrix} 1 & 1 \\ 1 & -1 \end{bmatrix} $$
We can see from the definition that $H$ takes a qubit and sends it to a superposition between $\ket{0},\ket{1}$. $H$ is unitary as $H^2 = I$:
$$H^2 = \frac{1}{2} \begin{bmatrix} 1 & 1 \\ 1 & -1 \end{bmatrix}^2 = \frac{1}{2}\begin{bmatrix} 2 & 0 \\ 0 & 2 \end{bmatrix} = I $$
Thus, $H$ also takes superpositions of $\ket{0},\ket{1}$ to single-state qubits.

We can directly calculate the effect the Hadamard gate has on $n$-qubits as such:

$$ H^{\otimes n} \ket{j_{n-1}...j_0} = \frac{1}{\sqrt{2^n}}\prod_{l=0}^{n-1} (\ket{0} + e^{i\pi j_l}\ket{1}) $$

where $j_i = \{0,1\}$ for $0 \leq i \leq n-1$.

Consequently, an \textit{equal} superposition where the probabilities for measuring a given $n$-qubit state are uniform:
$$H^{\otimes n}\ket{0...0} = \frac{1}{\sqrt{2^n}}\sum_{i=0}^{2^n-1}\ket{i}$$

This superposition is instrumental in many quantum algorithms such as Grover's algorithm to extract useful information from oracles.

\subsection{CNOT Gates}
Define the CNOT (Controlled-Not Gate) as the operator which takes a 2-qubit system sends them as follows:
\begin{gather*}
  \ket{0}\ket{0} \mapsto \ket{0}\ket{0} \\
  \ket{0}\ket{1} \mapsto \ket{0}\ket{1} \\
  \ket{1}\ket{0} \mapsto \ket{1}\ket{1} \\
  \ket{1}\ket{1} \mapsto \ket{1}\ket{0}
\end{gather*}
The qubit on the bottom (target qubit) flips in respect to the value of the top qubit (control qubit). The corresponding matrix representation of $CNOT: \mathbb{C}^2 \rightarrow \mathbb{C}^2$ is:
  $$CNOT = \begin{bmatrix}1 & 0 & 0 & 0 \\ 0 & 1 & 0 & 0 \\ 0 & 0 & 0 & 1 \\ 0 & 0 & 1 & 0 \end{bmatrix} $$

%  \Qcircuit @C=1em @R=.7em {
%    \qw & \ctrl{1} & \qw \\
%    \qw & \targ & \qw
%  }

We can extend this idea to create a set of \textit{generalized} CNOT gates with the following matrices:

\end{document}
