\documentclass[12pt]{article}
\setlength{\oddsidemargin}{0in}
\setlength{\evensidemargin}{0in}
\setlength{\textwidth}{6.5in}
\setlength{\parindent}{0cm}
\setlength{\parskip}{\baselineskip}

\usepackage{amsmath,amsfonts,amssymb,amsthm}
\usepackage{physics}
\usepackage{qcircuit}
\usepackage{subfiles,import}

\newtheorem{theorem}{Theorem}
\newtheorem{lemma}{Lemma}
\author{Edward Kim}

\newcommand{\papertitle}{Quantum Complexity Theory}

\begin{document}
\title{Notes on \papertitle}
\maketitle

\subsection*{Complexity Classes}

\begin{enumerate}
  \item ${\bf BQP}$ (Bounded Quantum Polynomial Time) is the complexity classes characterizing the languages which are decidable in quantum polynomial time. Formally speaking,
  \begin{gather*}
    L \in {\bf BQP} \text{ iff } \\
    \exists \{Q_n\}_{n \in \mathbb{N}} \text{ a polynomial-time generated circuit family s.t} \\
    x \in L \text{ then } {\bf Pr}[Q_{|x|}(x) \text{ accepts } x] \geq \frac{2}{3} \\
    x \not\in L \text{ then } {\bf Pr}[Q_{|x|}(x) \text{ accepts } x] \leq \frac{1}{3}
  \end{gather*}
  \item ${\bf BPP} \subseteq {\bf BQP}$
\end{enumerate}



\end{document}
