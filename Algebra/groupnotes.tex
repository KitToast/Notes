
\documentclass[12pt]{article}
\setlength{\oddsidemargin}{0in}
\setlength{\evensidemargin}{0in}
\setlength{\textwidth}{6.5in}
\setlength{\parindent}{0in}
\setlength{\parskip}{\baselineskip}

\usepackage{amsmath,amsfonts,amssymb}
\usepackage{amsthm}

\title{Your Project Title}

\begin{document}

Notes on Group Theory (Informal/Scratch)
Edward Kim

\hrulefill

\section*{Cyclic groups}

Theorem: 
\begin{enumerate}
\item An infinite cyclic group has exactly two generators (if $a$ is a generator, then $a^{-1}$ is the only other generator).
\item Let $G$ be a finite cyclic group of order $n$, and let $x$ be a generator. The set of generators of $G$ consists of those powers $x^v$ of $x$ such that $v$ is relatively prime to $n$.
\item Let $G$ be a cyclic group. and let $a,b$ be two generators. Then there exists an automorphism of $G$ mapping $a$ onto $b$. Conversely, any automorphism of $G$ maps a on some generator of $G$.
\item Let $G$ be a cyclic group with order $n$ and let $d$ divide $n$. Then there exists a unique subgroup $H$ with order $d$. (Converse of Lagrange Theorem)
\item Let $G_1$ and $G_2$ be cyclic groups of order $m$,$n$ respectively such that they are relatively prime. Then $G_1 \times G_2$ is cyclic. 
\item Let $G$ be a finite abelian group. If $G$ is not cyclic, then there exists a prime $p$ and a subgroup of $G$ isomorphic to $C \times C$, where $C$ is cyclic of order $p$.
\end{enumerate}

\begin{proof}
\begin{enumerate}
\item Any infinite cyclic group $G$ is isomorphic to $\mathbb{Z}$ by the mapping $\phi: \mathbb{Z} \rightarrow G$ that takes $\phi(1) = a$ and $\phi(-1) = a^{-1}$. Let $z \in \mathbb{Z}$ such that $z \neq 1,-1$. We know that the elements of the  subgroup generated by $z$ are exactly the integers of the form $nz$ where $n \in \mathbb{Z}$. Hence, $<z> \subset \mathbb{Z}$ is a strict inclusion and does not generate $\mathbb{Z}$. Hence, the image $\phi(<z>)$ is a subgroup which cannot contain all elements of $G$ as $\phi$ is an isomorphism. The assertion immediately follows.

\item  Since $v$ does not divide $n$,  $cv$ cannot be equal any multiple of n for $ 0 \leq c < n$. By Lagrange's Theorem, the cyclic subgroup generated by $x^v$ cannot be a proper subgroup as that would contradict our assumption. Hence, the order of the subgroup generated by $x^v$ must be $n$ i.e $G$.

\item Let $\phi: G \rightarrow G$ be the map such that $\phi(a) = b$. As $\phi(a^n) = b^n$, it follows that $\phi$ is an automorphism (the same can be said for the inverse). To prove the converse, let $\psi \in Aut(G)$. We prove the case for cyclic groups of finite order $n$. Let $\theta: \mathbb{Z}/n\mathbb{Z} \rightarrow G$ be the map $\theta([r]) = a^r$ where $a$ is a generator for $G$. By part 2 above, we know that $Aut(\mathbb{Z}/n\mathbb{Z})$ consists of mappings from generators to generators. Using $\theta$, one can define the isomorphism $\sigma:  Aut(G) \rightarrow Aut(\mathbb{Z}/n\mathbb{Z})$. Thus, 
$\psi = \theta \circ \sigma \circ \theta^{-1}$ which shows that ever automorphism must map $a$ to another generator. 
\end{enumerate}
\end{proof}



\section*{Sylow Subgroups}

We work with finite groups $G$. Let $H$ be a subgroup of $G$. We call $H$ a p-Sylow subgroup if the order of $H$ is $p^n$ for some prime number $p$ and $n$ is the largest integer such that $p^n$ divides the order of $G$. 
\par Here is a big question: Can a converse albeit weak be proved with certain number-theoretic properties on the order of a finite group?

Sylow's Theorems

\begin{enumerate}
\item If $H$ is a $p$-group, then it must be contained in some $p$-Sylow group
\item All $p$-Sylow groups are conjugate to each other.
\item The number of $p$-Sylow groups is 1 (mod) $p$
\end{enumerate}

The proof gives an example of how we can count the number of Sylow groups using the class formula.

Question: Can we build p-Sylow groups from certain $p$-groups? (Direct product,etc)?

\par Lemma: Let $G$ be a finite group and let $p$ be the smallest prime that divides the order of $G$. If $H$ is a subgroup such that its order is $p$, then $H$ must be normal.

\begin{proof}
 We begin by focusing on the normalizer of the group. Let $N(H)$ denote the normalizer of $H$. Since $$ (G:H) = (H:N(H))(N(H): H) = p$$, this leaves that $N(H) = G$ or $N(H) = H$ since one of the products has to equal to 1. If $N(H) = G$, then we are done. Otherwise, let $N(H) = H$. Let $G$ act on $H$ by conjugation. Then we can construct a homomorphism $G \rightarrow S_p$ by indexing the cosets of $G/H$ and mapping any $g \in G$ to it's action on the cosets by conjugation. This homomorphism is well-defined as one can easily verify as follows:
let $a \cong b$, then conjugating them results in in the following: $xax^{-1}xb^-1x^{-1} = ab^{-1} \in H$. Thus, cosets are mapped to cosets under conjugation. Now let $K$ be the kernel of this map. Since $N(H) \subset K$, by definition, $H \subset K$. By Lagrange's Theorem, $$ (G: K) = (G: H)(H:K) $$ and by our isomorphism theorem, $|(G:K)| = p!$. Since $(G:H) = p$, there exists another prime which divides $(p-1)! = |(H:K)|$. However, this prime is smaller than $p$ and another application of Lagrange's theorem shows that this prime must divide the order of $G$ which contradicts our assumption above.
\end{proof}

\end{document}