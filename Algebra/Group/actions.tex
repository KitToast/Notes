\documentclass[main.tex]{subfiles}

\begin{document}
\section{Group Actions}

\begin{definition}
A set $X$ is a G-set if the group $G$ acts on $X$ in the following manner:
\begin{enumerate}
\item $ex = x, x \in X$
\item $(g_1g_2)x = g_1(g_2x), x \in X$
\end{enumerate}
\end{definition}

If the set $X$ is finite with cardinality $n$, then there exists a homomorphism $\phi: G \rightarrow \Sigma_n$
where $\Sigma_n$ is the permutation group of $n$ elements. The homomorphism is defined by the action of $g$ on all $n$ elements in the set $X$. This observation leads us to the following theorem:

\begin{theorem} 
(Cayley's Theorem)
Let $G$ be a finite group of order $n$, then $G$ is isomorphic to a subgroup of $\Sigma_n$.
\end{theorem}

\begin{proof}
We begin by indexing the elements of $G$ by the index set of $\{1...n\}$ and each $g \in G$ maps to the permuatation element in $\Sigma_n$ based on its action on each element in $G$. The image of $\phi$ as defined above is a subgroup by definition. 
\end{proof}


\end{document}