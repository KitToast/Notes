\documentclass[main.tex]{subfiles}

\begin{document}
\section{Characteristic Groups}

Let $G$ be a group and $\phi \in Aut(G)$. Let $H$ be a subgroup of $G$. We say that $H$ is fixed by $\phi$ if $\phi(H) = H$. Restricting $\phi$ to $H$ yields an automorphism on $H$. A subgroup $H$ is said to be  a characteristic subgroup of $G$ if $H$ is fixed by all $\psi \in Aut(G)$. A simple example is the center of $G$ or $Z(G)$. Given any $\phi \in Aut(G)$ and $y \in G$ and $x \in Z(G)$:
$$\phi(x)y = \phi(x\phi^{-1}(y)) = \phi (\phi^{-1}(y)x) = y \phi(x)$$

\begin{remark}
It's clear that if $K$ is a characteristic subgroup of $H$ and $H$ is a characteristic subgroup of $G$, then
$K$ is a characteristic subgroup of $G$.
\end{remark}

\subsection{Commuter/Derived subgroup}

The commuter of elements $x,y \in G$ is defined as $[x,y] = xyx^{-1}y^{-1}$. The commuter(derived) subgroup is defined as the subgroup generated by commuters of all elements in $G$. Denote this subgroup as $G'$

\begin{lemma}
Let $G$ be a group. Then $G'$ is characteristic in $G$.
\end{lemma}

\begin{proof}
Given any $\phi \in Aut(G)$, $\phi([x,y]) = [\phi(x),\phi(y)]$. Thus, $G' \leq \phi(G')$. Conversely, since $\phi^{-1} \in Aut(G), \phi(G') \leq \phi^{-1}(\phi(G')) = G'$. Hence, $\phi(G') = G'$.
\end{proof}

\begin{theorem}
Let $G$ be a group and $N \unlhd G$. $G/N$ is abelian iff $G' \leq N$.
\end{theorem}

\begin{proof}
If $G/N$ is abelian, then for cosets $xN,yN \in G/N, xyN = yxN$. Thus, $x^{-1}y^{-1}xy \in N$. Conversely,
 since for any $x,y \in G$, $[xG',yG'] = [x,y]G'$, the commuter subgroup of $G/G'$ is trivial and hence abelian. Since $G'$ is normal, we utilize the factorization theorem to arrive at the following isomorphism,
 
$$(G/G')/(N/G') \cong G/N $$ 
 
Since $G/N$ is isomorphic to a quotient of an abelian group, $G/N$ is abelian as well.  
\end{proof}

Note that if $G$ is abelian, then $G'$ is the trivial subgroup which is contained in every subgroup in $G$.
\end{document}