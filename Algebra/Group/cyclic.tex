
\documentclass[main.tex]{subfiles}

\begin{document}
\section{Cyclic groups}

\begin{theorem}
\label{general} 	
	\begin{enumerate}
	\item An infinite cyclic group has exactly two generators (if $a$ is a generator, then $a^{-1}$ is the only other generator).
	\item Let $G$ be a finite cyclic group of order $n$, and let $x$ be a generator. The set of generators of $G$ consists of those powers $x^v$ of $x$ such that $v$ is relatively prime to $n$.
	\item Let $G$ be a cyclic group. and let $a,b$ be two generators. Then there exists an automorphism of $G$ mapping $a$ onto $b$. Conversely, any automorphism of $G$ maps a on some generator of $G$.
	\item Let $G$ be a cyclic group with order $n$ and let $d$ divide $n$. Then there exists a unique subgroup $H$ with order $d$. (Converse of Lagrange Theorem)
	\item Let $G_1$ and $G_2$ be cyclic groups of order $m$,$n$ respectively such that they are relatively prime. Then $G_1 \times G_2$ is cyclic. 
	\item Let $G$ be a finite abelian group. If $G$ is not cyclic, then there exists a prime $p$ and a subgroup of $G$ isomorphic to $C \times C$, where $C$ is cyclic of order $p$.
	\end{enumerate}

\end{theorem}

\begin{proof}
\begin{enumerate}
\item Any infinite cyclic group $G$ is isomorphic to $\mathbb{Z}$ by the mapping $\phi: \mathbb{Z} \rightarrow G$ that takes $\phi(1) = a$ and $\phi(-1) = a^{-1}$. Let $z \in \mathbb{Z}$ such that $z \neq 1,-1$. We know that the elements of the  subgroup generated by $z$ are exactly the integers of the form $nz$ where $n \in \mathbb{Z}$. Hence, $<z> \subset \mathbb{Z}$ is a strict inclusion and does not generate $\mathbb{Z}$. Hence, the image $\phi(<z>)$ is a subgroup which cannot contain all elements of $G$ as $\phi$ is an isomorphism. The assertion immediately follows.

\item  Since $v$ does not divide $n$,  $cv$ cannot be equal any multiple of n for $ 0 \leq c < n$. By Lagrange's Theorem, the cyclic subgroup generated by $x^v$ cannot be a proper subgroup as that would contradict our assumption. Hence, the order of the subgroup generated by $x^v$ must be $n$ i.e $G$.

\item Let $\phi: G \rightarrow G$ be the map such that $\phi(a) = b$. As $\phi(a^n) = b^n$, it follows that $\phi$ is an automorphism (the same can be said for the inverse). To prove the converse, let $\psi \in Aut(G)$. We prove the case for cyclic groups of finite order $n$. Let $\theta: \mathbb{Z}/n\mathbb{Z} \rightarrow G$ be the map $\theta([r]) = a^r$ where $a$ is a generator for $G$. By part 2 above, we know that $Aut(\mathbb{Z}/n\mathbb{Z})$ consists of mappings from generators to generators. Using $\theta$, one can define the isomorphism $\sigma:  Aut(G) \rightarrow Aut(\mathbb{Z}/n\mathbb{Z})$. Thus, 
$\psi = \theta \circ \sigma \circ \theta^{-1}$ which shows that ever automorphism must map $a$ to another generator. 
\end{enumerate}
\end{proof}

\subsection{Automorphisms of cyclic groups}

\begin{theorem}
$Aut(\mathbb{Z}_p) \cong \mathbb{Z}_{p-1}$
\end{theorem}

\begin{proof}
By \ref{general}, $Aut(\mathbb{Z}_p) \cong \mathbb{F}_p^{\times}$ which the field of mutlipicative elements of the finite field $\mathbb{P}$. Let $d | (p-1)$ and $F(X) = X^d - 1 = 0$ a polynomial of degree $d$ over $\mathbb{F}_p$. If $x \in \mathbb{F}_p^{\times}$, such that the order of $x$ divides $d$, the $x$ is a solution for $F$. On the other hand, if the cyclic group generated by $x$ has order $d$, then $x$ is a solution for $F$. As the number of distinct solutions for $F$ is bounded by $d$, it follows that the roots of the polynomial are generated by a single cyclic group generated by $x$. Hence, every element of order $d$  in $\mathbb{F}_p^{\times}$ must be contained in the cyclic group generated by $x$ which is isomorphic to $\mathbb{Z}_d$. Let $f_d$ be the number of elements of order $d$ in $\mathbb{F}_p^{\times}$. By the aforementioned isomorphism, $f_d \leq z_d$ where $z_d$ is the number of elements in $\mathbb{Z}_{p-1}$ of order $d$. By \ref{general} again, we note that all elements in $\mathbb{Z}_{p-1}$ of order $d$ is contained in a unique subgroup of order $d$ in $\mathbb{Z}_{p-1}$. Thus $z_d$ is equal to the number of elements of order $d$ in $\mathbb{Z}_d$.  Hence, the following equalities hold:
$$ \sum_{d | (p-1)} f_d = |\mathbb{F}_p^{\times} | = p -1 = |\mathbb{Z}_{p-1}| = \sum_{d|(p-1)} z_d $$
This forces the equality $f_d = z_d$. In particular, $f_{p-1} = z_{p-1} > 0$. Thus, by a properly mapping the elements in the equality above, we see that $$\mathbb{F}_{p}^{\times} \cong \mathbb{Z}_{p-1}$$    
\end{proof}

\begin{definition}
Let $<x> \cong Z_n$ and let $\sigma_m$ be the endomorphism such that $\sigma(x) = x^m$ such that $gcd(m,n) = 1$. Let $k$ be the smallest integer such that $(\sigma_{m})^k = x^{m^k}$ i.e $m^k \equiv 1 (\text{mod } n)$.
Let $\phi(n)$ denote Euler's Totient function. If $k = \phi(n)$ then we deem $m$ is a primitive root modulo $n$ 
\end{definition}

From \ref{general}, if $m$ is a primitive root modulo $n$ then $Aut(\mathbb{Z}_n)$ is cyclic. There is a more general theorem to recognize if $Aut(\mathbb{Z})_n$ is cyclic. The proof will be omitted for brevity.

\begin{theorem}
\label{cyclic}
$Aut(Z_n)$ is cyclic iff $n=2$ or $n=4$ or $n = p^k$ or $2p^k$ for some prime $p$ and $k \in \mathbb{N}$
\end{theorem}

We will show a simple example of the above theorem:

\begin{proposition}
$Aut(\mathbb{Z}_{p^2}) \cong \mathbb{Z}_{p^2 - p}$
\end{proposition}

\begin{proof}
Let $m$ be a primitive root modulo $p$. Then it suffices to prove that $m$ or $m+p$ are primitive roots modulo $p^2$. To proof this, note that $gcd(m,p^2) = 1$ for all $1 \leq m \leq p-1$. By the same argument, $gcd(m+p,p^2) = 1$ for $1 \leq m \leq p-1$ except for elements of the form $ap$ for $1 \leq a \leq p$. Thus, we see that $Aut(\mathbb{Z}_n)$ is cyclic and is of order $p^2 - p$ as desired.
\end{proof}

The proof above inspires an inductive argument for the proof of \ref{cyclic}. (More on that later).

\end{document}