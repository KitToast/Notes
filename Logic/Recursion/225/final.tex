\documentclass[11pt]{article}
\usepackage{amsmath,textcomp,amssymb,geometry,graphicx,enumerate,tikz,verbatim,relsize}
\def\Name{Edward Kim}  % Your name
\def\SID{}  % Your student ID number
\def\Homework{Final Project} % Number of Homework
\def\Session{Spring 2016}

\title{Math 225B--\Homework}
\author{\Name}
\markboth{Math 225B--\Homework\ \Name}{Math 225B--\Homework\ \Name}
\pagestyle{myheadings}
\newenvironment{qparts}{\begin{enumerate}[{(}a{)}]}{\end{enumerate}}
\def\endproofmark{$\Box$}
\newenvironment{proof}{\par{\bf Proof}:}{\endproofmark\smallskip}

\textheight=9in
\textwidth=6.5in
\topmargin=-.75in
\oddsidemargin=0.25in
\evensidemargin=0.25in
\setlength\parindent{0pt}
\DeclareMathOperator{\dis}{d}
\newcommand*{\R}{\mathbb{R}}

\begin{document}
\maketitle 

\section*{Problem 1}
\begin{proof}
 Following the hint provided, suppose we had a function $f$ such that $f$ maps $x$ to the least stage such that the approximation to $A$ is correct on all numbers $n$ such that $n < x$. Since $A$ is a $\Delta^0_2$ subset of $\mathbb{N}$, it follows that it can be computed from the halting set $\emptyset'$. We recall the halting set can compute the least stage $s$ where:
 \begin{gather*}
  n \in A \Leftrightarrow (\forall t \geq s) \text{ } g(n,t) = 1 \\
  n \not\in A \Leftrightarrow (\forall t \geq s) \text{ } g(n,t) = 0
 \end{gather*}

 Thus, $f(x)$ can retrieve this information directly from $A$ for all inputs $n \leq x$ and it follows that $f$ is computable from $A$. Now suppose, we have a function $g$ computable from a set $X$ such that $g$ dominates $f$ on all inputs $n \in \mathbb{N}$. However, we observe that $g$ computes $A$ by the following scheme:
 we compute $A(n)$ by running the program that approximates $A$ at input $n$ for $g(n+1)$ stages. In other words, let $y = f(n+1)$, then by definition of $f$:
 \begin{gather*}
  n \in A \Leftrightarrow (\forall t \geq y) \text{ } g(n,t) = 1 \\
  n \not\in A \Leftrightarrow (\forall t \geq y) \text{ } g(n,t) = 0
 \end{gather*}
 
 From $g(n+1) \geq f(n+1)$, it follows that running the program for $g(n+1)$ stages will result in the correct output. Thus, $A$ can be computed by $X$ through $g$. \end{proof}

\section*{Problem 2}
\begin{proof}
 We provide a priority construction where the negative requirements are injured finitely often. We give the requirements below:
 \begin{equation*} \label{posreq}
    P_e: \{e\}^D \neq G
 \end{equation*}
and 
\begin{equation*} \label{negreq}
    N_e: \{e\}^G \neq D
\end{equation*}

For the positive requirements, we hold on a potential witness $x$ and observe if $\{e\}^D(x) \downarrow = 0$ on stage $s$. If so, we enumerate $x$ into $G_{s+1}$. If not, no action is required since either $\{e\}^D\downarrow \neq 0$ or $\{e\}^D(x)\uparrow$.  
Hence, we say that $P_e$ requires attention at stage $s$ if for some potential witness $x$

\begin{equation*}
    \{e\}^{D_s}_s(x) \downarrow = 0
\end{equation*}

The negative requirements will require us to restrain values from being enumerated into $G$. Thus, we shall define the following length function:
\begin{equation*}
  L(e,s) = \max \{x: \forall n \leq x \text{ } \{e\}^{G_s}_s(n) \downarrow = D_s(n) \} 
\end{equation*}

We similarly define the restraint function $r(e,s)$ which prohibits any values smaller than the largest value used in the computation of all $\{e\}^{G_s}_s(n)\downarrow$ for some stage $s$ where $n \leq L(e,s)$.
Lastly, we give the priority list:
\begin{equation*}
 P_1, N_1, P_2, N_2, ...
\end{equation*}

and outline the construction: \newline \newline
\underline{Stage 0:} Let $G_0 = \emptyset$. For each positive requirement $P_e$, let the first picked potential witness be $e$. For each negative requirement $N_e$, let the restraint functions be initialized $r(e,0) = -1$. \newline
\underline{Stage s:} We find the least positive requirement $P_i$ that requires attention at this stage. We enumerate the witness $x$ into $G_s$. We require all postive requirements $P_j$ where $j \geq i$ to find new witnesses $x_j$ where 
$x_j \geq r(e,s)$ for $e \leq i$ and $x_j \not\in G_s$. \newline

We see that all of the positive requirements are eventually met. All there is left to verify is that our negative requirements are met. \newline

\underline{Claim:}  $\{e\}^G \neq D$ for all $e$. \newline
\begin{proof}
 Suppose that $\{e\}^G = D$ for some $e$. Then we see that $\lim_s L(e,s) = \infty$. Thus, for all input $n$, there must exist a stage $s$ such that $L(e,s) > n$ and $N_e$ is never injured after stage $s$. By definition of $L$, $\{e\}^G_s(n) \downarrow = D(n)$. Thus, $D$ is recursive which contradicts our original assumption.\end{proof}
\end{proof}

\section*{Problem 3}
\begin{proof}
 We use a priority argument where we enumerate $W$ one element $x$ per stage ($x \in W_{s+1} - W_s$). We only enumerate the element $x$ into one of the sets $A$ and $B$ and the elements must respect the negative requirements of higher priority.
 We outline the positive and negative criterion below:
 \begin{gather*}
  P: \text{For every $x \in W_{s+1} - W_s$, $x \in A_s$ or $x \in B_s$} \\
  N_{[e,A]}: \{e\}^A \neq D \\
  N_{[e,B]}: \{e\}^B \neq D
 \end{gather*}

 Let us define $L^A(e,s)$ as we did in Problem 2. That is $L^A(e,s)$ the largest input $x$ in which the computations of $\{e\}^A_s$ and $D_s$ agree on inputs $n \leq x$. We define the restraint function $r^A(e,s)$ simiarly as in Problem 2 as well. Functions $L^B(e,s)$ and $r^B(e,s)$ are defined similarly with $B$ in place of $A$. \newline
 
 \underline{Stage 0:} $A = \emptyset$, $B = \emptyset$ \newline
 \underline{Stage s:} Let $x \in W_{s+1} - W_s$. Choose the highest negative priority $N_{[e,i]}$ where $i = A,B$ and enumerate $x$ on into the other set. In other words, 
 if $i = A$, then enumerate $x$ into $B_{s+1}$ and if $i = B$, then enumerate $x$ into $A_{s+1}$. \newline
 
 We prove that every negative requirement is injured finitely many times and that $r^A$ and $r^B$ converge through induction. For any $[e,i]$, let us assume that after stage $s'$, all priorities $[g,j] < [e,i]$ are never injured and $r^A(g)$ and $r^B(g)$ converge for stages $s \geq s'$. Let us find a stage $t$ where $N_{[e,i]}$ is never injured after stage $t$. To find this, we have to ensure that
 $B$ does not enumerate anymore elements after this stage into any of the restraints of higher priority. This ensures that no elements enumerated from $B$ can injure $N_{[e,i]}$ since all of the higher priority restraints are completely settled.
 Take a value $r > \max \{r^j(g) \mid [g,j] < [e,i]\}$. Let $t$ be the stage such that $B_t \restriction r = B \restriction r$. As mentioned previously, $N_{[e,i]}$ cannot be 
 injured after this point and it follows that $r^A(e)$ and $r^B(e)$ converge through the proof of the claim in Problem 2.
\end{proof}


\section*{Problem 4}
\begin{proof}
 We will first conclude that every branch in the aforementioned tree must split into two infinite branches. We shall proceed by contradiction and assume that there exists an isolated branch $B$ in $T$. We recursively compute $B$ as follows: let 
 $\sigma \subset B$. We can compute the next branch $B$ will take by computing strings of $T$ longer than $|\sigma|$ and extending $\sigma$. Let $\sigma*0$ or $\sigma*1$ represent strings $\sigma$ concatenated with 0 and 1 respectively. Since $B$ is an isolated branch, there must exist a level $L$ such that all strings extending one of the branches must eventually terminate within a finite amount of stages. The next step of the path will traverse the branch which has not terminated within the level $L$. 
 However, this implies that $f$ is a recursive path in $T$ which contradicts our original assumption about the infinite paths in $T$. Thus, every branch in $T$ must split into two infinite branches. Thus, given a set $X \subseteq \mathbb{N}$, we can trace a path $P$ in $T$ such that if a real number $z_i$ codes a set recursive in $X$, then $z_i \not\subset P$ (follows from the observation that only a countable number of these types of real numbers can exist). Hence, this path $P$ is an infinite path such that $X \not\geq_T P$. We give a simple argument for the uncountability of the infinite paths in $T$. Suppose that there were a countable number of infinite paths in $T$, then we can code these paths into a countable set of real numbers $X'$.
 However, by the result above, there must exist an infinte path $P'$ in $T$ such that $X' \not\geq_T P'$. However, by construction, $P'$ cannot be any of the infinite paths in $X'$ which contradicts the definition of $X'$. Hence, the uncountability of the infinite paths in $T$ follows.
\end{proof}

\section*{Problem 5}
\begin{proof}
 The intuition behind the proof of this proposition is to create two sets $A$ and $B$ such that $\{e\}^A \neq X$ and $\{e\}^B \neq X$ for all $e$ but the join $A \oplus B$ encodes enough information about $X$ to ensure that $X \leq_T A \oplus B$.
 We provide the following finite extension construction by constructing the characteristic function of $A$ and $B$ through finite initial segments $\{f_{s}\}_{s \in \omega}$ and $\{g_{s}\}_{s \in \omega}$ respectively: \newline 
 
 \underline{Stage 0:} $f_{A_0} = \emptyset$, $f_{B_0} = \emptyset$. \newline \newline
 \underline{Stage $s+1 = 2e$:} We shall ensure that $\{e\}^{A} \neq X$. We proceed by finite extension and query the following $\Sigma^0_1$ sentence to a $\emptyset'$-oracle  :
  \begin{equation}
    (\exists \sigma)(\exists \tau) (\exists x)(\exists s)(\exists v)(\exists w)[f_s \subset \sigma,\tau \text{ \& } \{e\}^{\sigma}_s(x) = y \text{ \& } \{e\}^{\tau}_s(x) = z \text{ \& } y \neq z]
  \end{equation}

  This ensures that at least one of the finite extensions $\sigma,\tau$ disagrees with $X$ on witness $x$. We then use a $X$-oracle to find if $\{e\}^{\sigma}_s(x) \neq X(x)$ or $\{e\}^{\tau}_s(x) \neq X(x)$. Without the loss of generality, we then let $\sigma$ be the extension that disagrees with $X$. In addition, we extend $g_s$ by copying the $|\sigma| - |f_{s}|)$ bits that were added to $f_{s}$. This can be recursively done by
  appending $\sigma(n)$ for $|f_{s}|+1 \leq n \leq |\sigma|$ to the end of $g_s$ in order. We then code the necessary information into $A$ and $B$ to code $X$. Let $n = |\sigma|$. If $X(e) = 0$, then let $f_{s+1}(n) = 0$ and $g_{s+1}(n) = 1$. If $X(e) = 1$, then let $f_{s+1}(n) = 1$ and $g_{s+1}(n) = 0$. By construction, we see that both $f_{s+1}$ and $g_{s+1}$ are of the same length. \newline
  
  \underline{Stage $s+1 = 2e+1$:} We shall now ensure that $\{e\}^{B} \neq X$. Repeat the above steps except replace  $f$ with $g$ and $A$ with $B$. \newline
  
  \underline{Claim:} $A \oplus B \leq_T X$
  \begin{proof}
    This claim follows from our construction procedure above. Each nonzero stage queries a $\emptyset'$-oracle and a $X$-oracle. By our assumption that $X \geq_T \emptyset'$, we see that the construction is done $X$-recursively.
  \end{proof} \newline
  
  \underline{Claim:} $X \leq_T A \oplus B$
  \begin{proof}
    We see that, by construction, the differences between $A$ and $B$ code the characteristic function for $X$. We can recursively check for any difference between $A$ and $B$. If a difference is detected at index $n$, we consider two cases. If $A(n) = 1$, then $X(n) = 1$. If $B(n) = 1$, then $X(n) = 0$.
  \end{proof}   
  
\end{proof}
\end{document}